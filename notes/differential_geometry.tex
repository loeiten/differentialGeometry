\documentclass[a4paper, 12pt]{article}
\usepackage{../../../LaTeX/general}
\title{\vspace{-8ex}A short note about differential geometry\vspace{-1ex}}
\author{Michael L{\o}iten}
\date{\vspace{-2ex}\today}

\pagestyle{fancy}
\fancyhf{}
\lhead{Differential geometry}
\chead{}
\rhead{\today}
\cfoot{\thepage}
\renewcommand{\sectionmark}[1]{\markright{#1}{}}

%===================================================
\everymath{\displaystyle}
\usepackage{titlesec}
%\titleformat*{\section}{\bfseries}%\vspace{-0.2cm}}


% Section without numbering
%\renewcommand\thesection{\hspace{-0.5cm}}
%===================================================

\renewcommand{\arraystretch}{0.8}

% References
\def\divurl{http://physicspages.com/index-physics-relativity/dinverno-introducing-einsteins-relativity-problems/}
\def\cotan{https://www.physicsforums.com/threads/basis-for-tangent-space-and-cotangent-space.283108/}


\begin{document}
\maketitle

\section{Manifold}
A \emph{manifold} is a geometric object that consist of a collection of points 
(that is a locus of points), each point satisfying the condition that the area 
of the manifold near that point resembles Euclidean (flat space). The number $n$ 
of dimensions needed to define a manifold is the number of parameters needed to 
locate all the points on it.\\
\\
Example:\\
A sphere in $\mathbb{R}^3$ is a 2D manifold. It needs two parameters to 
trace out the surface (all the points of the manifold). We can write the circle 
as
%
\begin{align}
 &x = r\sin(\theta)\cos(\phi) \label{eq:x}\\
 &y = r\sin(\theta)\sin(\phi) \label{eq:y}\\
 &z = r\cos(\theta) \label{eq:z}
\end{align}
%
where $\theta \in [0,\pi)$ and $\phi \in [0,2\pi)$ are the parameters. The 
sphere is non-Euclidean, but small enough parts of it is. In constraint form 
the equation can be written as
%
\begin{align*}
 x^2+y^2+z^2-r^2 = 0
\end{align*}
%
If we have $m$ dimensions in the subsurface embedded in an $n$-dimensional 
manifold, we must have $n-m$ constraints.






\section{Coordinates} \label{sec:coord}
$x$, $y$ and $z$ in equation (\ref{eq:x} - \ref{eq:z}) are called 
\emph{coordinates}. A coordinate system is a set of $n$ variables which fix a 
geometric object.

Each point on a $n$-dimensional manifold possesses a set of 
$n$ coordinates. Coordinate systems which covers only a part of the manifold is 
called a \emph{coordinate patch}. A set of coordinate patches that covers the 
whole manifold is called an \emph{atlas} The theory of manifolds tells us how 
to get from one coordinate patch to another by a coordinate transformation in 
the overlap region.

As seen from equation (\ref{eq:x} - \ref{eq:z}), the coordinates 
are given in terms of three other variables $r$, $\theta$ and $\phi$, which in 
themselves form a different coordinate system. We have
%
\begin{align*}
 &r = \sqrt{x^2+y^2+z^2}\\
 &\theta = \arccos\L(\frac{z}{\sqrt{x^2 + y^2 + z^2}}\R)\\
 &\phi = \arctan\L(\frac{y}{x}\R)
\end{align*}
%
In general the change of coordinates $x^a \to x'^a$ is given by
%
\begin{align}
 (x')^a = f^a(x^1,x^2,\ldots,x^n)\qquad a\in{1,2,\ldots,n} \label{eq:transform}
\end{align}
%
where the $f$'s are single valued, continuous differentiable function for 
certain ranges of their arguments. Notice that the superscript in equation 
(\ref{eq:transform}) does not refer to taking powers, but rather to components 
of a contravariant tensor, and that the dangerous notation $(x')^a = 
x^a(x^1,x^2,\ldots,x^n)$ often is used.






\section{Change of coordinates} \label{sec:change_of_coordinates}
If we know how the bases relative to each other, for example
%
\begin{align}
 &\ve{a}_1=\frac{1}{\sqrt{2}}(\ve{b}_1 + \ve{b}_2)&
 &\ve{a}_2=3\ve{b}_2
 \label{eq:eq_with_basis_vector}
\end{align}
%
We can write
%
\begin{align*}
 [\ve{v}]_b &= A_{a\to b}[\ve{v}]_a\\
 %
 %
 \begin{bmatrix}
  v_{1,b}\\
  v_{2,b}
 \end{bmatrix}_b
 &=
  \begin{bmatrix}
  [\ve{a}_1]_b & [\ve{a}_2]_b
 \end{bmatrix}_{a\to b}
   \begin{bmatrix}
  v_{1,a}\\
  v_{2,a}
 \end{bmatrix}_a\\
 %
 %
 \begin{bmatrix}
  v_{1,b}\\
  v_{2,b}
 \end{bmatrix}_b
 &=
  \begin{bmatrix}
  \frac{1}{\sqrt{2}} & 0\\
  \frac{1}{\sqrt{2}} & 3
 \end{bmatrix}_{a\to b}
   \begin{bmatrix}
  v_{1,a}\\
  v_{2,a}
 \end{bmatrix}_a\\
 %
 %
 v_{1,b} &= \frac{1}{\sqrt{2}}v_{1,a}
 \numberthis \label{eq:coord_v1b}\\
 v_{2,b} &= \frac{1}{\sqrt{2}}v_{1,a} + 3v_{2,a} 
 \numberthis \label{eq:coord_v2b}
\end{align*}
%
The example here was in rectilinear coordinates (so there is no difference 
between co- and contravariant tensors), but in general (and for contravariant 
vectors), we have that%
\footnote{We will here use index notation (which means that the individual 
tensors in an expression gets individual indices) with the Einstein summation 
rule (when an index variable appears twice in a single term (called a dummy 
index) it implies summation of that term over all the values of the index).}
%
\begin{align}
 A_{\text{unprimed}\to \text{primed}}
 =
 \begin{bmatrix}
  \parti{(x')^i}{x^j}
 \end{bmatrix}
 \label{eq:contravariant_matrix}
\end{align}
%
Where equation (\ref{eq:contravariant_matrix}) is a matrix with the $x$'s 
referring to the different coordinates. The \emph{Jacobian} is 
defined as $J\defined|A_{\text{unprimed}\to \text{primed}}|$. A sufficient 
(but not necessary) requirement for having a well defined transformation is 
that $J$ has to be finite in the point under consideration. Note that the matrix
%
$\begin{bmatrix}
 \parti{x^i}{(x')^j}
\end{bmatrix}$
has the determinant $1/J$.

\vspace{0.5cm}
\begin{greenbox}{Mnemonic: Lowest index = row number in $A_{x \to x'}$}
A way to remember that the lowest index denotes the row 
number (that is the index $^j$ in equation (\ref{eq:contravariant_matrix})), is 
to observe that rows are located under each other, and that the lowest index 
(index $^j$ in equation (\ref{eq:contravariant_matrix})) is 
located under the upper index (index $^i$ in equation 
(\ref{eq:contravariant_matrix})). 
\end{greenbox}

\subsection{Example: Transformation of vector field from cylindrical to 
Cartesian coordinate system}
Let's assume we have a (contravariant) vector given in the basis $\{\hv{\rho}, 
\hv{\phi}, \hv{z}\}$, and would like to express it in the Cartesian basis 
$\{\hv{x}, \hv{y}, \hv{z}\}$. The transformation equations (the coordinates 
written as a function of the other coordinates) read
%
\begin{align*}
 &x = r \cos\phi&
 &r = \sqrt{x^2+y^2}&
 \\
 &y = r \cos\phi&
 &\phi = \atan\L(\frac{y}{x}\R)&
 \\
 &z=z&
 &z=z&
\end{align*}
%
Note that
%
\begin{align}
 \ve{A} =
 A_x \hv{x} + A_y \hv{y} + A_z \hv{z} =
 A_\rho \hv{\rho} + A_\phi \hv{\phi} + A_z  \hv{z}
 \label{eq:component}
\end{align}
%
So if we know how the $\hv{x}$, $\hv{y}$ and $\hv{z}$ can be written in terms 
of $\hv{\rho}$, $\hv{\phi}$ and $\hv{z}$, we can simply write the components 
in the new coordinate system to complete the transformation.

Let us try to use (\ref{eq:contravariant_matrix}). We get
%
\begin{align*}
 _{\{\hv{x}, \hv{y}, \hv{z}\}} \ve{A}
  =
  \L[\parti{(x, y, z)}{(\rho, \phi, z)}\R] \;
   _{\{\ve{\rho}, \ve{\phi},\ve{z}\}}\ve{A}
  = \begin{bmatrix} \cos\phi & -\rho\sin\phi & 0 \\
                    \sin\phi &  \rho\cos\phi & 0 \\
                    0 & 0 & 1
 \end{bmatrix} \;
 _{\{\ve{\rho}, \ve{\phi},\ve{z}\}}\ve{A}
\end{align*}
%
If we try to transform $\ve{\phi}$, we get
%
\begin{align*} 
  \begin{bmatrix} \cos\phi & -\rho\sin\phi & 0 \\
                    \sin\phi &  \rho\cos\phi & 0 \\
                    0 & 0 & 1
 \end{bmatrix}
 \begin{bmatrix} 0 \\
		 1 \\
		 0 \\
 \end{bmatrix}
 =
 \begin{bmatrix} -\rho\sin\phi \\
		 \rho\cos\phi \\
		 0 \\
 \end{bmatrix}
\end{align*}
%
Meaning that 
%
\begin{align*}
 \ve{\phi} = -\rho\sin\phi \hv{x} + \rho\cos\phi \hv{y}
\end{align*}
%
We see that this vector has length $\rho$, but we wanted to transform from 
something written in a unit basis (where all the basis vectors has length $1$). 
To ensure this we must normalize the columns in the transformation matrix to 
one, in other words, we must use
%
\begin{align*}
  \L[
     \frac{\displaystyle \parti{(x, y, z)}{\phi}}{
           \displaystyle \L\|\parti{(x,y, z)}{\phi}\R\| } \R]
\end{align*}
%
for each column in the transformation matrix. We will use the notation
%
\begin{align*}
  \L[
     \frac{\displaystyle \parti{(x, y, z)}{(\rho, \phi, z)}}{
           \displaystyle \L\|\parti{(x,y, z)}{(\rho, \phi, z)}\R\| } \R]
\end{align*}
%
to denote that all the columns in the transformation matrix is normalized. With 
this notation, we finally get that
%
\begin{align*}
 _{\{\hv{x}, \hv{y}, \hv{z}\}} \ve{A}
  =
    \L[
     \frac{\displaystyle \parti{(x, y, z)}{(\rho, \phi, z)}}{
           \displaystyle \L\|\parti{(x,y, z)}{(\rho, \phi, z)}\R\| } \R] \;
   _{\{\hv{\rho}, \hv{\phi},\hv{z}\}}\ve{A}
  = \begin{bmatrix} \cos\phi & -\sin\phi & 0 \\
                    \sin\phi &  \cos\phi & 0 \\
                    0 & 0 & 1
 \end{bmatrix} \;
 _{\{\hv{\rho}, \hv{\phi},\hv{z}\}}\ve{A}
\end{align*}



\section{Co- and contravariant vectors}
A basis is spanning the space and needs (in general) to be evaluated at each 
point in the space. It can be thought of as a set of reference axes.
A change of scale on the reference axes corresponds to a change of units in the 
problem. For instance, in changing scale from meters to centimeters (that is, 
dividing the scale of the reference axes by $100$), the components of a 
measured velocity vector will multiply by $100$.

For a vector (such as a direction vector or velocity vector) to be 
basis-independent, the components of the vector must contra-vary with a change 
of basis to compensate. The components of vectors (as opposed to those of dual 
vectors) are said to be contravariant. Examples of vectors with contravariant 
components include the position of an object relative to an observer, or any 
derivative of position with respect to time.

For a dual vector (also called a convector) to be basis-independent, the 
components of the dual vector must co-vary with a change of basis to remain 
representing the same convector. Examples of covariant vectors generally appear 
when taking a gradient of a function.





\subsection{Contravariant tensors}
A contravariant vector (that is a tensor of rank $1$) is defined by a vector 
which transforms as
%
\begin{align*}
 (A')^i = \parti{(x')^i}{x^j} A^j
\end{align*}
%
A higher rank tensor will transform as
%
\begin{align*}
 (A')^{ab\ldots} = \parti{(x')^{a}}{x^{c}}\parti{(x')^{b}}{x^{d}}\ldots 
A^{cd\ldots}
\end{align*}
%

\vspace{0.5cm}
\begin{greenbox}{Mnemonic: Contravariant components $= \; \uparrow$}
 The indices in the components are up, and the primed quantity is on 
top in the partial differential, since the ``n'' in ``contravariant'' can be 
thought of as a arrow pointing up. The primed quantities are written with the 
same index.
\end{greenbox}
%
\subsubsection{Why the coordinates in equation (\ref{eq:transform}) are with 
superscript}
Consider two neighboring points $P$ and $Q$ on a manifold with coordinates 
$x^a$ and $x^a + \text{d}x^a$. The infinitesimal vector $\ve{PQ} = 
\text{d}x^a$, is attached to $P$. The components in another coordinate system 
$x'^a=x'^a(x^a)$ can be found by using the chain rule
%
\begin{align}
 \text{d}(x')^a = \parti{(x')^a}{x^b}\text{d}x^b \label{eq:infinitesimal_vec}
\end{align}
%
and hence are components of equation (\ref{eq:transform}) written 
contravariantly. Note that the vectors in equation (\ref{eq:infinitesimal_vec}) 
are nothing but a specials cases of general vectors, and that one can always 
find the transformation by using the chain rule.



\subsection{Covariant tensors}
Covariant vectors are defined by transforming as
%
\begin{align*}
 (A')_i = \parti{x^j}{(x')^i} A_j
\end{align*}
%
A higher rank tensor will transform as
%
\begin{align*}
 (A')_{ab\ldots} = \parti{x^{c}}{(x')^{a}}\parti{x^{d}}{(x')^{b}}\ldots 
A_{cd\ldots}
\end{align*}
%
Note that differentiation of $\phi=\phi(x^b(x'))$ with respect to $(x')^a$ 
gives the contravariant transformation matrix as 
$\parti{\phi}{(x')^a}=\parti{x^b}{(x')^a}\parti{\phi}{x^b}$ due to the chain 
rule. Note that it is always possible to use the chain rule in order to get 
the transformation\label{foot:phi}.

\vspace{0.5cm}
\begin{greenbox}{Mnemonic: Covariant components $= \; \downarrow$}
 The indices in the components are down, and the primed quantity is 
 in the lower part of the partial differential, since the ``v'' in 
 ``covariant'' can be thought of as an arrow pointing downwards. Note 
 that the coordinates in the transformation matrix still have an upper 
 index. The primed quantities are written with the same index.
\end{greenbox}


\subsection{Mixed tensors}
A mixed tensor transforms as
%
\begin{align*}
 (A')^{ab\ldots}_{\qquad cd\ldots} = 
 \parti{(x')^{a}}{x^{e}}\parti{(x')^{b}}{x^{f}}\ldots 
 \parti{x^{g}}{(x')^{c}}\parti{x^{h}}{(x')^{d}}\ldots 
  A^{ef\ldots}_{\qquad gh\ldots}
\end{align*}
%
Note the notation, and that $A^{ab\ldots}_{\qquad cd\ldots} \neq 
A^{\qquad ab\ldots}_{cd\ldots}$ as $\ve{e}^a\ve{e}_c \neq \ve{e}_a\ve{e}^c$ in 
general, and that tensors such as $A^{a\ldots \quad c\ldots}_{\quad b\ldots}$ 
exist.

At last, it is worth mentioning that any tensor equation (such as 
$X_{ab}=Y_{ab}$) holds true in \textbf{every} coordinate system.




\section{Pseudotensors}
Up until this point we have only considered polar vectors, which transformed to 
its negative under inversion of its coordinate axes. There is another type of 
vectors called \emph{pseudovectors} or \emph{axial} vectors which are invariant 
under inversion. Whereas polar tensors transforms as
%
\begin{align*}
 \ve{A}' = \ve{S}\ve{A}
\end{align*}
%
pseudotensors transforms as
%
\begin{align*}
 \ve{A}' = \det(\ve{A})\ve{S}\ve{A}
\end{align*}
%
Examples of pseudovectors include angular velocity vector $\ve{\omega}$, 
angular momentum $\ve{L}$ and torque $\ve{\tau}$. All of these are defined 
from a cross product. We have that
%
\begin{align*}
 \text{vector} \times \text{vector} &= \text{pseudovector}\\
 \text{pseudovector} \times \text{pseudovector} &= \text{pseudovector}\\
 \text{pseudovector} \times \text{vector} &= \text{vector}\\
 \text{vector} \times \text{pseudovector} &= \text{vector}\\
\end{align*}
%
However $\ve{B}$ is also a pseudovector. One can convince oneself of that by 
knowing that $\ve{v}$ and $F$ is a polar vector when looking at the Lorentz 
force%
\footnote{Note: The sum of a polar vector with a pseudovector is neither a 
polar vector nor a pseudovector, and behaves rather strange under rotation. 
Such vectors can occur, for example when calculating decay due to the weak 
interaction.}%
.




\section{Basis vectors and metrics}
In general a vector can be written as
%
\begin{align*}
 \ve{A} = A^i\ve{e}_i = A_i\ve{e}^i
\end{align*}
%
And a second rank tensor can be written in four different ways
%
\begin{align*}
 \ve{A} = 
 A^{ij}\ve{e}_i\ve{e}_j =
 A_{ij}\ve{e}^i\ve{e}^j =
 A^{i}_{\;\;j}\ve{e}^i\ve{e}_j =
 A_{i}^{\;\;j}\ve{e}_i\ve{e}^j
\end{align*}
%
Where the covariant basis (transforms covariantly) can be written as
%
\begin{align}
 \ve{e}_i = \parti{}{x^i} \label{eq:cov_basis}
\end{align}
%
and the contravariant basis (transforms contravariantly) can be written as
%
\begin{align}
 \ve{e}^i = \text{d}x^i \label{eq:con_basis}
\end{align}
%
(see figure \ref{fig:basis}).

\vspace{0.5cm}
\begin{greenbox}{Mnemonic: $\parti{}{x^i} = $ covariant,  $\text{d}x^i = $ 
contravariant }
 One can remember that $\parti{}{x^i}$ is the \emph{covariant} unit vector 
and that $\text{d}x^i$ is the \emph{contravariant} unit vector by realizing 
that the index in $\text{d}x^i$ is located above the index in $\parti{}{x^i}$, 
and therefore we can use that the ``n'' in ``contravariant'' is ``pointing up'' 
as described above.
\end{greenbox}


\subsection{Basis vectors in differential form}\label{sec:bvdf}
The basis vectors written in equation (\ref{eq:cov_basis}) and 
(\ref{eq:con_basis}) surely doesn't look like the ``classical'' basis vectors 
one encountered when first facing linear algebra. They are rather defined as 
operators. 

It actually makes sense if one consider the \emph{tangent space}. Let us 
consider a point $P$ on our manifold (for example the sphere in the example in 
section \ref{sec:coord}). If we draw all possible curves (for example 
the curve $S(r,\theta,\phi)$ parametrized by $t$ [that is 
$S(r,\theta,\phi) = S(r(t),\theta(t),\phi(t))$]) through $P$ and take the 
directional 
derivative $\Bigg($that is $\deri{S}{t} = \parti{S}{r}\parti{r}{t} + 
\parti{S}{\theta}\parti{\theta}{t} + \parti{S}{\phi}\parti{\phi}{t}\Bigg)$ of 
each curve, we can generate a collection of tangents to $P$, which is called 
the tangent space. Notice that the partial derivative $\parti{S}{x^i}$ at 
the point is the change of the line along the coordinate $x^i$. Thus the tuple 
of all $\parti{S}{x^i}$ $\Bigg($in our case $\L\{\parti{S}{r}, 
\parti{S}{\theta}, \parti{S}{\phi}\R\} \Bigg)$ are along the direction 
which spans the tangent space (see figure \ref{fig:basis}). This is true for all 
curves $\in C^{\infty}$ (that is, smooth curves). Notice that the partial 
derivative of the curve along the coordinate direction transforms covariantly 
(as shown in (\ref{foot:phi})), and is independent of the curve $S$.

A tangent vector $\ve{X}$ (a vector living in the tangent space) can thus be 
written in operator form as $\ve{X}=a^i\parti{}{x^{i}}$. The dual space to 
the tangent space is called the \emph{cotangent space}. The cotangent space is 
made up of all linear functionals (a function from a vector space into its 
underlying scalar field [that is it takes a vector in the vector space as an 
input and returns a scalar value]) on the tangent space. The i'th coordinate 
function of the tangent space is defined by $\d x^i(\ve{X}) = a_i$, and thus $\d 
x^i$ makes up the dual (cotangent space) to the tangent space. It can be shown 
that the cotangent basis vector in a point of a coordinate is perpendicular on 
the surface spanned by holding the coordinate under consideration fixed to the 
point of consideration while varying all the other coordinates. This is 
depicted 
in figure \ref{fig:basis}.
%
\begin{figure}[h!]
\center
 \includegraphics[width=0.5\textwidth]{figures/co-contra}
 \caption{Basis vectors in a point.}
 \label{fig:basis}
\end{figure}

\vspace{0.5cm}
\begin{greenbox}{Mnemonic: How to change unit vectors}
 Covariant unit vectors: Insert a function of a curve as a function of the new 
 coordinates $(x')$ (which again can be expressed in the current coordinates 
 $x$), for example the function $f((x')(x))$, and use the chain rule on the 
 expression on $\parti{f((x')(x))}{x^i}$.\\
 Contravariant unit vectors: Use the chain rule of the infinitesimal.
\end{greenbox}



\section{The Kronecker delta and metrics}
The Kronecker delta $\delta^i_j = 1$ if $i=j$, and zero otherwise. The 
independence of the coordinates can be expressed as
%
\begin{align*}
 \delta^i_j = \parti{x^i}{x^j} = \parti{x^i[(x')^k\{x^l\}]}{x^j} = 
 \parti{x^i}{(x')^k}\parti{(x')^k}{x^j},
\end{align*}
%
where $x^i$ is a function of $(x')^k$, which again is a function of $x^l$, here 
written as $x^i = x^i[(x')^k\{x^l\}]$

A mnemonic of the function is to say that in order to remove a delta, all the 
variables with the same index as the upper index must be transformed to the 
lower index.

Further, from section \ref{sec:bvdf} we have that
%
\begin{align*}
 \d x^i \L(\parti{}{x^j}\R) = \ve{e}^i\cdot\ve{e}_j = \delta^i_j
\end{align*}
%
The metric tensors are defined as
%
\begin{align*}
 &g_{ij} = \ve{e}_i\cdot\ve{e}_j&
 &g^{ij} = \ve{e}^i\cdot\ve{e}^j
\end{align*}
%
and we notice that
%
\begin{align*}
 &g^{ij} g_{jk} = \delta^i_j&
 &e_i = g_{ij}e^j &
 &e^i = g^{ij}e_j&
\end{align*}




\section{The $\nabla$ operator}
The del operator is simply defined as
%
\begin{align*}
 \nabla = \ve{e}_i \parti{}{x_i}
\end{align*}
%
from this we get that products such as $(\nabla \ve{B}) \cdot \hv{b}$ can be 
written as
%
\begin{align*}
 (\nabla \ve{B}) \cdot \hv{b} = 
 \ve{e}_i \parti{}{x_i} (B^j \ve{e}_j) \cdot \frac{B_k}{\|\ve{B}\|}\ve{e}^k
\end{align*}
%
Note that there exists tensor identities just like vector identities, which is 
coordinate independent and makes it easy to evaluate expressions such that the 
one above. We have that
%
\begin{align*}
 (\nabla \ve{B}) \cdot \hv{b} = \grad \|\ve{B}\|
\end{align*}




\section{A transformation example}
Let's say that we would to go from a description in Cartesian coordinates 
(given by the coordinates $\ve{R}$), to a description in curvilinear (given by 
the coordinates $\ve{u}$). We have that the component in the curvilinear 
coordinates can be written like $(v')^i = 
(v')^i(\ve{R}(\ve{u}))$, where 
$\ve{R}(\ve{u}) =
 \begin{bmatrix}
  x^1(u^1,u^2,u^3)& 
  x^2(u^1,u^2,u^3)&
  x^3(u^1,u^2,u^3)
 \end{bmatrix}
$
and that the components written in the Cartesian coordinates can be written like
$v^i = 
v^i(\ve{u}(\ve{R}))$, where 
$\ve{u}(\ve{R}) =
 \begin{bmatrix}
  u^1(x^1,x^2,x^3)& 
  u^2(x^1,x^2,x^3)&
  u^3(x^1,x^2,x^3)
 \end{bmatrix}
$. The transformation from curvilinear to Cartesian coordinates can be written 
like
%
\begin{align*}
 \ve{v} &=
 (v')^i(\ve{e}')_i =
 (v')^i\parti{}{R^i} = 
 (v')^i \parti{u^j}{R^i}\parti{}{u^j} =
 v^i \parti{u^j}{R^i}\parti{}{u^j} =
 v^i \parti{u^j}{R^i}\ve{e}_j
 \\
 \ve{w} &=
 (w')_i(\ve{e}')^i = 
 (w')_i\text{d}R^i = 
 (w')_i \parti{R^i}{u^j}\text{d}u^j =
 w_i \parti{R^i}{u^j}\text{d}u^j =
 w_i \parti{R^i}{u^j}\ve{e}^j
\end{align*}
%
Notice that the notation can appear somewhat confusing. We have that a 
vector $\ve{a}$ is represented contravariantly as $(a')^i(\ve{e}')_i$ in the 
primed basis and as $(a')^i \parti{R^i}{u^j} \ve{e}_j$ in the unprimed 
basis, which makes $(a')^i = a^i$. 

Further note that some authors prefer the somewhat bad notation $\ve{e}_i = 
\grad u^i$, which translates to: ``The Cartesian basis vector written 
contravariantly in $u$-coordinates'' (the notation misses the last 
$\parti{}{u^j}$ in order to be correct), and $\ve{e}^i = \parti{\ve{R}}{u^i}$ 
translates to: ``The Cartesian basis vector written covariantly in 
$u$-coordinates'' (the notation misses the last $\text{d}u^j$ in order to be 
correct).

%http://physics.stackexchange.com/questions/57754/what-is-a-dual-cotangent-space

\nocite{diverno_1992}
\nocite{arfken_2012}
\nocite{DHaeseleer_1991}
\nocite{wiki_co_cont}
\nocite{wiki_co_trans}
\nocite{diverno_url}
\nocite{physics_forum}
\nocite{wiki_cot_space}
\nocite{mathworld_pseudo}

\markboth{Bibliography}{}
\bibliographystyle{../../../LaTeX/prsty}
\bibliography{bib/localbib}
\end{document}
